%%%%%%%%%%%%%%%%%%%%%%%%%%%%%%%%%%%%%%%%%
% Wenneker Article
% LaTeX Template
% Version 2.0 (28/2/17)
%
% This template was downloaded from:
% http://www.LaTeXTemplates.com
%
% Authors:
% Vel (vel@LaTeXTemplates.com)
% Frits Wenneker
%
% License:
% CC BY-NC-SA 3.0 (http://creativecommons.org/licenses/by-nc-sa/3.0/)
%
%%%%%%%%%%%%%%%%%%%%%%%%%%%%%%%%%%%%%%%%%

%----------------------------------------------------------------------------------------
%	PACKAGES AND OTHER DOCUMENT CONFIGURATIONS
%----------------------------------------------------------------------------------------

\documentclass[10pt, a4paper, twocolumn]{article} % 10pt font size (11 and 12 also possible), A4 paper (letterpaper for US letter) and two column layout (remove for one column)

%%%%%%%%%%%%%%%%%%%%%%%%%%%%%%%%%%%%%%%%%
% Wenneker Article
% Structure Specification File
% Version 1.0 (28/2/17)
%
% This file originates from:
% http://www.LaTeXTemplates.com
%
% Authors:
% Frits Wenneker
% Vel (vel@LaTeXTemplates.com)
%
% License:
% CC BY-NC-SA 3.0 (http://creativecommons.org/licenses/by-nc-sa/3.0/)
%
%%%%%%%%%%%%%%%%%%%%%%%%%%%%%%%%%%%%%%%%%

%----------------------------------------------------------------------------------------
%	PACKAGES AND OTHER DOCUMENT CONFIGURATIONS
%----------------------------------------------------------------------------------------

\usepackage[english]{babel} % English language hyphenation

\usepackage{microtype} % Better typography

\usepackage{amssymb,amsmath} % Math packages for equations

\usepackage[svgnames]{xcolor} % Enabling colors by their 'svgnames'

\usepackage[hang, small, labelfont=bf, up, textfont=it]{caption} % Custom captions under/above tables and figures

\usepackage{booktabs} % Horizontal rules in tables

\usepackage{lastpage} % Used to determine the number of pages in the document (for "Page X of Total")

\usepackage{graphicx} % Required for adding images

\usepackage{enumitem} % Required for customising lists
\setlist{noitemsep} % Remove spacing between bullet/numbered list elements

\usepackage{sectsty} % Enables custom section titles

\usepackage{bm}

\usepackage{algorithm}
%\usepackage{algorithmic}
\usepackage[noend]{algpseudocode}

\allsectionsfont{\usefont{OT1}{phv}{b}{n}} % Change the font of all section commands (Helvetica)

%----------------------------------------------------------------------------------------
%	MARGINS AND SPACING
%----------------------------------------------------------------------------------------

\usepackage{geometry} % Required for adjusting page dimensions

\geometry{
	top=1cm, % Top margin
	bottom=1.5cm, % Bottom margin
	left=2cm, % Left margin
	right=2cm, % Right margin
	includehead, % Include space for a header
	includefoot, % Include space for a footer
	%showframe, % Uncomment to show how the type block is set on the page
}

\setlength{\columnsep}{7mm} % Column separation width

%----------------------------------------------------------------------------------------
%	FONTS
%----------------------------------------------------------------------------------------

\usepackage[T1]{fontenc} % Output font encoding for international characters
\usepackage[utf8]{inputenc} % Required for inputting international characters

\usepackage{XCharter} % Use the XCharter font

%----------------------------------------------------------------------------------------
%	HEADERS AND FOOTERS
%----------------------------------------------------------------------------------------

\usepackage{fancyhdr} % Needed to define custom headers/footers
\pagestyle{fancy} % Enables the custom headers/footers

\renewcommand{\headrulewidth}{0.0pt} % No header rule
\renewcommand{\footrulewidth}{0.4pt} % Thin footer rule

\renewcommand{\sectionmark}[1]{\markboth{#1}{}} % Removes the section number from the header when \leftmark is used

%\nouppercase\leftmark % Add this to one of the lines below if you want a section title in the header/footer

% Headers
\lhead{} % Left header
\chead{\textit{\thetitle}} % Center header - currently printing the article title
\rhead{} % Right header

% Footers
\lfoot{} % Left footer
\cfoot{} % Center footer
\rfoot{\footnotesize Page \thepage\ of \pageref{LastPage}} % Right footer, "Page 1 of 2"

\fancypagestyle{firstpage}{ % Page style for the first page with the title
	\fancyhf{}
	\renewcommand{\footrulewidth}{0pt} % Suppress footer rule
}

%----------------------------------------------------------------------------------------
%	TITLE SECTION
%----------------------------------------------------------------------------------------

\newcommand{\authorstyle}[1]{{\large\usefont{OT1}{phv}{b}{n}\color{DarkRed}#1}} % Authors style (Helvetica)

\newcommand{\institution}[1]{{\footnotesize\usefont{OT1}{phv}{m}{sl}\color{Black}#1}} % Institutions style (Helvetica)

\usepackage{titling} % Allows custom title configuration

\newcommand{\HorRule}{\color{DarkGoldenrod}\rule{\linewidth}{1pt}} % Defines the gold horizontal rule around the title

\pretitle{
	\vspace{-30pt} % Move the entire title section up
	\HorRule\vspace{10pt} % Horizontal rule before the title
	\fontsize{32}{36}\usefont{OT1}{phv}{b}{n}\selectfont % Helvetica
	\color{DarkRed} % Text colour for the title and author(s)
}

\posttitle{\par\vskip 15pt} % Whitespace under the title

\preauthor{} % Anything that will appear before \author is printed

\postauthor{ % Anything that will appear after \author is printed
	\vspace{10pt} % Space before the rule
	\par\HorRule % Horizontal rule after the title
	\vspace{20pt} % Space after the title section
}

%----------------------------------------------------------------------------------------
%	ABSTRACT
%----------------------------------------------------------------------------------------

\usepackage{lettrine} % Package to accentuate the first letter of the text (lettrine)
\usepackage{fix-cm}	% Fixes the height of the lettrine

\newcommand{\initial}[1]{ % Defines the command and style for the lettrine
	\lettrine[lines=3,findent=4pt,nindent=0pt]{% Lettrine takes up 3 lines, the text to the right of it is indented 4pt and further indenting of lines 2+ is stopped
		\color{DarkGoldenrod}% Lettrine colour
		{#1}% The letter
	}{}%
}

\usepackage{xstring} % Required for string manipulation

\newcommand{\lettrineabstract}[1]{
	\StrLeft{#1}{1}[\firstletter] % Capture the first letter of the abstract for the lettrine
	\initial{\firstletter}\textbf{\StrGobbleLeft{#1}{1}} % Print the abstract with the first letter as a lettrine and the rest in bold
}

%----------------------------------------------------------------------------------------
%	BIBLIOGRAPHY
%----------------------------------------------------------------------------------------

\usepackage[backend=bibtex,style=authoryear,natbib=true]{biblatex} % Use the bibtex backend with the authoryear citation style (which resembles APA)

\addbibresource{example.bib} % The filename of the bibliography

\usepackage[autostyle=true]{csquotes} % Required to generate language-dependent quotes in the bibliography
 % Specifies the document structure and loads requires packages

%----------------------------------------------------------------------------------------
%	ARTICLE INFORMATION
%----------------------------------------------------------------------------------------

\title{Una Metaheurística GRASP para Integración No Orientada en Grafos} % The article title

\author{
	\authorstyle{Manuel Dubinsky\textsuperscript{1}, 
	César Massri\textsuperscript{2}, Fernando Asteasuain\textsuperscript{1}} % Authors
	\newline\newline % Space before institutions
	\textsuperscript{1}\institution{Dpto. Tecnología y Administración, Ingeniería Informática, Universidad Nacional de Avellaneda, Argentina}\\ % Institution 1
	\textsuperscript{2}\institution{Dpto. de Matemática, FCEyN, Universidad de Buenos Aires, Argentina}
}

% Example of a one line author/institution relationship
%\author{\newauthor{John Marston} \newinstitution{Universidad Nacional Autónoma de México, Mexico City, Mexico}}

\date{\today} % Add a date here if you would like one to appear underneath the title block, use \today for the current date, leave empty for no date

%----------------------------------------------------------------------------------------

\begin{document}

\maketitle % Print the title

\thispagestyle{firstpage} % Apply the page style for the first page (no headers and footers)

%----------------------------------------------------------------------------------------
%	ABSTRACT
%----------------------------------------------------------------------------------------

\begin{abstract}
Dado un grafo conexo $G=(V,E)$ y una función de pesos en los nodos $x: V 
\rightarrow \mathbb{R}$, definimos el \textit{diferencial no orientado} 
$dx: E \rightarrow \mathbb{R}^+$ en cada eje $e=(v,w)$ como $dx(e) = 
|x(v) - x(w)|$. En el presente trabajo abordamos el siguiente problema: 
dado un grafo $G=(V,E)$ y una función de pesos en los ejes $f: E 
\rightarrow \mathbb{R}^+$, se trata de encontrar la función de pesos en 
los nodos $x: V \rightarrow \mathbb{R}$ cuyo \textit{diferencial no 
orientado} ajuste del mejor modo posible (en términos de cuadrados 
mínimos) los pesos de los ejes dados por $f$. Para ello presentamos un 
algoritmo basado en la metaheurística GRASP cuya simplicidad permite 
paralelizarlo para procesar instancias de tamaño arbitrario.
\end{abstract}

%----------------------------------------------------------------------------------------
%	ARTICLE CONTENTS
%----------------------------------------------------------------------------------------

\section{Introducción}

\subsection{Descripción del problema}

En esta sección se presenta el problema que abordaremos en el trabajo. 
De acuerdo a nuestro entendimiento, se trata de una formulación 
novedosa, por lo tanto no hay antecedentes a los cuales podamos 
remitirnos.

\smallskip

Dado un grafo conexo $G=(V,E)$ y una función de pesos en los ejes 
$f: E \rightarrow \mathbb{R}^+$, queremos hallar una función de pesos 
en los nodos $x: V \rightarrow \mathbb{R}$ de modo de minimizar el 
error:

$$E(x) := \sum_{e \in E} (dx(e) - f(e))^2$$

donde $dx: E \rightarrow \mathbb{R}^+$ es el \textit{diferencial 
no orientado} de $x$ y se define en cada eje $e=(v,w)$ como $dx(e) = 
|x(v) - x(w)|$.

\smallskip

Por otro lado, es posible definir en un grafo dirigido una noción de 
\textit{diferencial orientado} del siguiente modo: dado un grafo 
dirigido $\vec G=(\vec V,\vec E)$ y una función de pesos en los nodos 
$\vec x: \vec V \rightarrow \mathbb{R}$, se define el 
\textit{diferencial orientado} $\vec{dx}: \vec E \rightarrow \mathbb{R}$ 
en cada eje $e=v \rightarrow w$ como $\vec{dx}(e) = \vec x(w) - 
\vec x(v)$.

Es importante notar que un \textit{diferencial no orientado} induce 
naturalmente un \textit{diferencial orientado}, informalmente: hay que 
orientar cada eje del nodo ``menor" al ``mayor". Recíprocamente, si
se analizan todas las orientaciones posibles del grafo original, es 
posible resolver el problema propuesto. Concretamente, se trata de 
encontrar por un lado la ``mejor" \textit{matriz de incidencia 
dirigida} $D$, y por el otro la mejor función de pesos en 
los ejes $x: V \rightarrow \mathbb{R}$ de modo de minimizar la 
expresión:

$$E(x) = ||D\bm{x}-\bm{f}||^2$$

donde:

$$
\bm{x} = 
\begin{bmatrix}
	x(v_1)\\
	\dots \\
	x(v_n)
\end{bmatrix}, 
\bm{f} = 
\begin{bmatrix}
	f(e_1)\\
	\dots \\
	f(e_m)
\end{bmatrix}
$$

El algoritmo que proponemos justamente explora el espacio de todas las 
orientaciones posibles del grafo original.

\subsection{Aplicaciones}

Las aplicaciones son aquellas en las cuales necesitamos construir una 
función de los nodos que respete ciertas restricciones diferenciales.

Por ejemplo, supongamos que los nodos representan regiones geográficas 
y queremos construir un mapa topográfico basado en ciertas relaciones 
diferenciales de altura entre algunas regiones.

Otro ejemplo podría ser el de ordenar cronológicamente ciertos eventos 
en base a algunas relaciones diferenciales entre ellos.

%------------------------------------------------

\section{Preliminares}

\textbf{La \textit{matriz de incidencia dirigida}}

\smallskip

La \textit{matriz de incidencia dirigida} ($D$) de un grafo dirigido 
$\vec G = (\vec V, \vec E)$ es una matriz de $m \times n$ (donde $|\vec
 E| = m$ y $|\vec V| = n$) tal que para cada eje orientado 
$e_k=v_i \rightarrow v_j$: $D_{ki} = -1$, $D_{kj} = 1$ y vale $0$ en 
las demás entradas de la fila $k$.

\smallskip

En otros términos, una matriz de incidencia dirigida representa una 
posible orientación de los ejes de un grafo no dirigido $G$. Y el 
espacio de matrices de incidencia dirigida asociadas a $G$ 
equivale al espacio de todas las orientaciones posibles de $G$.

\bigskip

\noindent{\textbf{La \textit{matriz laplaciana}}}

La \textit{matriz laplaciana} ($L$) de un grafo $G$ es una matriz de 
$n \times n$ definida del siguiente modo:

$$
	L_{ii} =
	\begin{cases}
	deg(v_i) & \text{si $i = j$} \\
	-1 & \text{si existe el eje $(v_i,v_j)$} \\
	0 & \text{en otro caso} 
	\end{cases}
$$

El rango de $L$ es $n-c$ donde $c$ es la cantidad de componentes 
conexas. En particular si $G$ es conexo, el rango de $L$ es $n-1$

\bigskip

\noindent{\textbf{Relación entre las matrices $L$ y $D$}}

Es fácil verificar que las matrices $L$ y $D$ se relacionan de acuerdo
a la expresión:

$$L = D^t D$$

Es importante notar que $L$ es independiente de cualquier orientación 
que se le asocie al grafo. Y que el rango de $D$ es el mismo que el de 
$L$. Como en nuestro caso $G$ es conexo el rango de $D$ es $n-1$.

\bigskip

\noindent{\textbf{Sistema lineal inducido por $D$}}

Recordemos que el problema que queremos resolver es encontrar la matriz 
de incidencia dirigida $D$, es decir la ``mejor" orientación de los ejes 
del grafo $G$, de modo de minimizar la expresión:

$$E(x) = ||D\bm{x}-\bm{f}||^2$$

Para eso será preciso explorar el espacio de matrices de 
incidencia dirigida asociadas a $G$. Una vez que conseguimos una 
matriz candidata, el problema de minimización es equivalente a encontrar
dónde se anula el gradiente de $E(x)$:
 
$$\nabla E = [\frac{\partial E}{\partial x_1}, \dots, \frac{\partial 
E}{\partial x_n}] = D^tDx-D^tv=0$$

Que a su vez es equivalente a resolver el sistema lineal:

$$D^tDx = Lx = D^tv$$

Como el rango de $L$ es $n-1$, el sistema puede no tener solución. 
Mediante métodos iterativos es posible converger al punto más cercano a 
una solución. Específicamente en este trabajo emplearemos el método de 
\textit{Gradiente Conjugado}.

\smallskip

Es importante mencionar que en el último 
tiempo ha habido grandes avances en la creación de nuevos métodos 
iterativos para resolver esta clase particular de sistemas lineales 
(aquellos que involucran una matriz laplaciana) denominados 
\textit{sistemas laplacianos}.

\bigskip

\noindent{\textbf{Norma de la proyección ortogonal}}

Para elaborar el criterio que nos permita comparar dos matrices de 
incidencia dirigida y decidir cuál es mejor en el contexto del problema 
que queremos resolver, vamos a referirnos a nociones básicas de 
proyección ortogonal de un vector sobre un subespacio de un espacio 
vectorial.

\smallskip

Sean $v,w \in \mathbb{R}^n$, notamos $p_v(w)$ a la proyección de $w$ 
sobre $v$. Mediante relaciones trigonométricas básicas podemos 
calcular $||p_v(w)||$ (la norma de $p_v(w)$):

$$\frac{v w}{||w||} = \frac{cos(\alpha) \ ||v|| \ ||w||}{||w||} = 
cos(\alpha) \ ||v|| = ||p_v(w)||$$

De modo más general, notamos $p_S(w)$ a la proyección ortogonal de $w$
sobre un subespacio $S \subset \mathbb{R}^n$ generado por una base 
ortogonal $S = <s_1, \dots, s_k>$. La norma de $p_S(w)$ puede calcularse
de modo análogo:

$$
||p_S(w)|| = ||\begin{bmatrix}
	||p_{s_1}(w)|| \\
	\dots \\
	||p_{s_k}(w)||
\end{bmatrix}||
$$

\smallskip

Para dar una intuición geométrica del hecho precedente, supongamos (sin 
pérdida de generalidad) que $S = <e_1, \dots, e_k>$ está dada por los 
primeros $k$ vectores canónicos y $w = (w_1, \dots, w_n)$. 
Claramente la proyección ortogonal de $w$ sobre $S$ es 
$(w_1, \dots, w_k)$ (las primeras $k$ componentes de $w$). Entonces 
verifiquemos la expresión anterior:

$$
||\begin{bmatrix}
	||p_{s_1}(w)|| \\	
	\dots \\
	||p_{s_k}(w)||
\end{bmatrix}|| = 
||\begin{bmatrix}
	cos(\alpha_1) \ ||w|| \\
	\dots \\
	cos(\alpha_k) \ ||w||
\end{bmatrix}|| = 
||\begin{bmatrix}
	w \ e_1 \\
	\dots \\
	w \ e_k
\end{bmatrix}|| =  
||\begin{bmatrix}
	w_1 \\
	\dots \\
	w_k
\end{bmatrix}||
$$

\smallskip

Informalmente, el caso general se puede reducir a este caso simple 
mediante una matriz rotación, que preserva las normas.

\bigskip

\noindent{\textbf{Metaheurísticas GRASP}}

En el contexto del tratamiento de problemas de complejidad $NP$, el 
enfoque metaheurístico procura encontrar buenas soluciones (no 
necesariamente las mejores) en tiempo polinomial. Algunas 
metaheurísticas son:

\begin{itemize}
	\item Simulated-annealing
	\item Tabu-search
	\item GRASP
	\item Algoritmos genéticos
	\item Colonia de hormigas
\end{itemize}

Encontrar la metaheurística más adecuada depende del problema que se 
quiera resolver y debe ser contrastada empíricamente con los resultados 
obtenidos por otras metaheurísticas.

\smallskip

Específicamente una metaheurística de tipo \textit{GRASP} combina dos 
aspectos. Por un lado, requiere cierto tipo de búsqueda local, es decir: 
explorar un entorno de la solución de la iteración actual, para lo cual 
es preciso definir una noción de vecindad en el espacio; en nuestro caso 
se tratará de definir una noción de vecindad en el espacio de matrices 
de incidencia dirigida asociadas a un grafo. Y por otro lado, GRASP 
requiere cierto grado de azar en la elección del vecino a explorar en 
la siguiente iteración; esto busca impedir que el algoritmo se 
estabilice entorno a un mínimo local (no necesariamente un mínimo 
global). 

\smallskip

En la próxima sección se presentará el algoritmo y se describirán las 
particularidades de la implementación de una metaheurística GRASP en el 
contexto del problema que queremos resolver.

\section{Implementación}

En esta sección describiremos el algoritmo y los detalles de la 
implementación.

\subsection{La ``mejor" matriz de incidencia dirigida}

El primer paso de la estrategia que vamos a desarrollar es tratar de 
encontrar la ``mejor" matriz de incidencia dirigida, o dicho en otros 
términos: de orientar los ejes del grafo $G$ del mejor modo posible para 
minimizar el error:

$$E(x) = ||D\bm{x}-\bm{f}||^2$$
 
Como el espacio matrices de incidencia dirigida asociadas a $G$ es un 
espacio de tamaño exponencial respecto a la cantidad de ejes, dado que 
cada eje puede tener dos orientaciones, nuestra propuesta está centrada 
en construir un algoritmo que devuelva la ``mejor solución posible".

\bigskip

\noindent{\textbf{Preprocesamiento de las hojas del grafo}}

Recordemos la formulación inicial del problema: queremos hallar una 
función de pesos en los nodos $x: V \rightarrow \mathbb{R}$ de modo de 
minimizar el error:

$$E(x) := \sum_{e \in E} (dx(e) - f(e))^2$$

Consideremos una hoja $v$ de $G$ (un nodo de grado 1), a su único 
vecino $w$ y al eje que los liga $e=(v,w)$. De acuerdo a la formulación, 
surge que $x(v) = x(w) \pm f(e)$, es decir que $x(v)$ queda 
determinado por $x(w)$, $f(e)$ y la elección de una orientación de $e$.

\smallskip

Por lo tanto, es posible reducir el grafo original $G$, eliminando 
recursivamente sus  hojas de modo de obtener una grafo más simple (un 
grafo con menos ejes y nodos):

\begin{verbatim}
         while (G has leaves) {
             G = stripLeaves(G);
         }
\end{verbatim}

Un corolario es que si $G$ es un árbol, entonces $x: V 
\rightarrow \mathbb{R}$ queda determinada por alguna orientación de 
los ejes y por $x(v)$ para un nodo $v$.

\smallskip

De ahora en adelante consideraremos que el grado de los nodos de $G$ es 
al menos 2.
 
\bigskip

\noindent{\textbf{Una base semi-ortogonal}}

Un hecho simple que se deriva de la expresión que vincula a las matrices 
$L$ y $D$:

$$L = D^t D$$

es que el producto interno de dos columnas distintas de una matriz de 
incidencia orientada vale $-1$ o $0$ dependiendo de si existe un eje 
entre los nodos asociados a dichas columnas o no. Notar el hecho 
evidente de que si el producto es $0$, las columnas son ortogonales.

\smallskip

Ahora notemos lo siguiente: si G es conexo, el ángulo entre dos 
columnas $d_i$ y $d_j$ de $D$ está contenido en el rango 
$[\frac{\pi}{2},\frac{2\pi}{3}]$. Para ver esto, supongamos que existe 
un eje entre los nodos $v_i$ y $v_j$ asociados a dichas columnas. 
El ángulo puede medirse de este modo:

$$cos(\alpha) \ ||d_i|| \ ||d_j||= d_i  d_j \implies cos(\alpha) = 
\frac{d_i d_j}{||d_i|| \ ||d_j||}$$

El módulo de esta fracción se maximiza cuando las normas de las columnas
 son más pequeñas o de modo equivalente: cuando los grados de los nodos 
 correspondientes se minimizan. Recordemos que el grado de cada nodo de 
 $G$, luego de preprocesar las hojas, es a lo sumo 2. Supongamos que  
 $deg(v_i) = deg(v_j) = 2$. Entonces resulta que:

$$cos \ \alpha = \frac{d_i d_j}{||d_i|| \ ||d_j||} = \frac{-1}{2}$$

\smallskip

Implicando que $\alpha$ es a lo sumo $\frac{2\pi}{3}$.

\smallskip

Denotaremos base \textit{semi-ortogonal} a una base (de un 
$\mathbb{R}$-espacio vectorial) tal que el ángulo entre cualquier par de 
generadores está en el rango $[\frac{\pi}{2},\frac{2\pi}{3}]$.

\smallskip

El hecho de que las columnas de una matriz de incidencia dirigida 
definan una base semi-ortogonal va a justificar el criterio que 
emplearemos para compararlas.

\bigskip

\noindent{\textbf{La \textit{pseudo-norma}. Un criterio para comparar matrices 
de incidencia dirigida}}

Una interpretación geométrica del problema $argmin_D||D \bm x- \bm f
||^2$ es la siguiente: encontrar la ``mejor" matriz de incidencia 
dirigida $D$ tal que la proyección ortogonal de $f$ sobre el subespacio 
generado por las columnas de $D$ tenga norma máxima. En el caso de que 
el vector $f$ esté contenido en el subespacio, entonces el problema 
tiene solución exacta, es decir: el sistema lineal $Dx = f$ tiene 
solución.

\smallskip

Para resolver el problema, necesitamos un criterio para comparar  
matrices de incidencia dirigida asociadas al grafo $G$. Como las 
columnas de una matriz de incidencia dirigida $D$ definen una base 
semi-ortogonal, aproximaremos la norma de la proyección 
ortogonal de $f$ sobre $D$ como si las columnas de $D$ fueran 
ortogonales y denotaremos a esta aproximación \textit{pseudo-norma}.
 De este modo, entre dos matrices de incidencia dirigida $D$ y 
 $\bar{D}$ nuestro criterio será optar por aquella con mayor 
 pseudo-norma. 

\smallskip

La pseudo-norma se puede calcular en dos pasos. El paso inicial sería 
calcular el producto interno de $f$ con las columnas de $D$:

$$
D^t f = \begin{bmatrix}
	cos(\alpha_1) \ ||d_1|| \ ||f||\\
	\dots \\
	cos(\alpha_n) \ ||d_n|| \ ||f||
\end{bmatrix}
$$

\smallskip

Y luego dividir cada componente por la norma correspondiente:

$$
pseudoNorm_f(D) = D^t f
\begin{bmatrix}
	||d_1||^{-1}\\
	\dots \\
	||d_n||^{-1}
\end{bmatrix} = \begin{bmatrix}
	cos(\alpha_1) \ ||f||\\
	\dots \\
	cos(\alpha_n) \ ||f||
\end{bmatrix}
$$

\smallskip

Notemos que para cambiar la dirección de un eje en una matriz de 
orientación dirigida $D$, hay que invertir la fila correspondiente al 
eje que se quiere modificar. Y de modo más general, si queremos cambiar 
la dirección de varios ejes simultáneamente la nueva matriz $\bar{D}$
se puede calcular del siguiente modo:

$$\bar{D} = \bar{I} D$$

\smallskip

donde $\bar{I}$ es la matriz diagonal de $m \times m$ que tiene un $-1$ 
en cada fila asociada a cada eje que queremos invertir y $1$ en el resto 
de sus componentes.

\smallskip

Notemos que la norma de las columnas de $D$ y $\bar{D}$ son iguales 
porque sólo difieren en el signo de algunas componentes. Por lo tanto, 
la norma de las columnas de las matrices de incidencia dirigida puede 
calularse una sola vez.

\bigskip

\subsection{El enfoque metaheurístico}

Recordemos que el problema que queremos resolver es el de encontrar una 
matriz de incidencia dirigida $D$ que maximice $pseudoNorm_f(D)$. 
Cabe aclarar que según nuestro entendimiento, no es posible resolverlo 
por métodos exactos porque el tamaño del espacio de matrices de 
incidencia dirigida es $2^m$  (donde 
$|E|=m$). Este espacio es extremadamente grande incluso para grafos 
pequeños. Por lo tanto, el problema requiere un enfoque metaheurístico 
que nos permita encontrar una solución ``buena" (aunque no 
necesariamente la mejor).

\smallskip

Como ya mencionamos, las metaheurísticas \textit{GRASP} combinan azar y 
búsqueda local. La búsqueda local se refiere a la exploración de 
una vecindad de un punto del espacio. En nuestro caso el espacio es el 
conjunto de matrices de incidencia dirigida asociadas a $G$. Una 
definición obvia de vecindad de una matriz de incidencia dirigida $D$ es 
el conjunto de matrices que difieren en la dirección de un eje, o sea 
aquellas que tienen una fila invertida con respecto a $D$. Cada vez que 
se encuentra un mínimo (o máximo), lo conservamos. La sencillez de 
GRASP y la posibilidad de paralelizarlo de un modo natural, lo 
convierten en una alternativa razonable para nuestro problema.

\subsection{El algoritmo}

La estrategia que presenta nuestro algoritmo es una variante en la que 
invertimos los aspectos de azar y búsqueda local de \textit{GRASP}: 
primero elegimos al azar un conjunto de vecinos y luego definimos como 
nuevo punto a explorar al mejor de ellos (el que tenga mayor 
pseudo-norma). 

\smallskip

El algoritmo está controlado por dos parámetros:

\begin{itemize}
	\item La vecindad de un punto está determinada por el parámetro 
	$\alpha$, que establece la cantidad de ejes del punto que deben 
	modificarse al azar para construir un vecino.
	\item La cantidad de vecinos del punto que debemos explorar en cada 
	iteración está dada por el parámetro $\beta$
\end{itemize}

A continaución presentamos el pseudo-código del algoritmo y luego 
comentamos las funciones auxiliares.

\begin{algorithm}
    \caption{Integrate($G,f,\alpha,\beta$)}
	\begin{algorithmic}
		\State $currentPoint \gets Initialize(G)$
		\State $bestPoint \gets currentPoint$
		\While{True}
			\State $\Omega \gets Neighbors(currentPoint)$
			\State $bestNeighbor \gets BestNeighbor(\Omega)$
			\If{PN(bestNeighbor,f) > PN(bestPoint,f)}
				\State $bestPoint \gets bestNeighbor$
			\EndIf
			\State $currentPoint \gets bestNeighbor$
		\EndWhile
	\end{algorithmic}
\end{algorithm}

\newpage
La función \textbf{Initialize(G)} se encarga de la fase de 
inicialización, que esencialmente consiste en construir 
una matriz de incidencia dirigida. En nuestro caso, orientamos los ejes 
del nodo menor al mayor de acuerdo a un etiquetado previo de los nodos 
del grafo. 


\section{Resultados}

\section{Trabajo futuro}

%----------------------------------------------------------------------------------------
%	BIBLIOGRAPHY
%----------------------------------------------------------------------------------------

\begin{thebibliography}{99} % Bibliography - this is intentionally simple in this template

\bibitem[Figueredo and Wolf, 2009]{Figueredo:2009dg}
Figueredo, A.~J. and Wolf, P. S.~A. (2009).
\newblock Assortative pairing and life history strategy - a cross-cultural
  study.
\newblock {\em Human Nature}, 20:317--330.
 
\end{thebibliography}

%----------------------------------------------------------------------------------------

\end{document}
